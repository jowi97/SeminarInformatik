In der Praxis hat sich jegliche Art von Cyber-Erpressungs-Malware auf IT-Systeme 
als sehr erfolgreich und lukrativ erwiesen. Jedes Jahr erzielen Erpresser so 
mehrere Milliarden US-Dollar Lösegeld und die Tendenz ist weiter steigend \cite[vgl.]{C..24.05.2020}. 
Durch die zunehmende Vernetzung und Digitalisierung entstehen immer größere 
potenzielle Angriffsziele, welche Opfer eines Ransomware-Angriffs werden können. 
Der eigene Computer, Unternehmens-IT-Systeme, SmartHome-Geräte oder IoT-Geräte – 
alle sind der Gefahr ausgesetzt. Nur ein vernetztes Gerät, welches täglich von 
Milliarden von Menschen benutzt wird, ist bis zum jetzigen Standpunkt noch kein 
Opfer von Ransomware-Angriffen geworden – das Auto. Gerade im Automobilbereich 
spielt das Thema Vernetzung und Digitalisierung eine sehr große Rolle und man 
verfolgt das Ziel eines vollständig vernetzten Straßenverkehrs an. Ein Auto muss 
mittlerweile mehr können als nur fahren und besteht aus sehr vielen softwaregesteuerten 
Komponenten. Diese wiederum erhöhen somit auch die potenzielle Angriffsfläche, vor 
allem für Ransomware. 
\newline
Der wissenschaftliche Artikel \cite{M.Wolf.2017} befasst sich mit diesem Thema und diskutiert 
einen solchen Angriff sowohl in der Theorie als auch in der Praxis und stellt Präventiv- 
als auch Gegenmaßnahmen dar. Der Inhalt dieses Artikels wird im Folgenden zusammengefasst 
wiedergegeben. 
