Das Wort „Ransomware“ kommt von dem englischen Wort „ransom“ und 
bedeutet übersetzt „Lösegeld“. Dabei handelt es sich um Schadprogramme, 
meistens Erpressungssoftware oder Verschlüsselungstrojaner, mit denen ein 
Angreifer den Computer sperren und darauf befindliche Daten verschlüsseln 
kann. Erst nachdem das Opfer einer Lösegeldzahlung nachkommt, werden die 
Daten wieder freigegeben. 
\newline
Im Gegensatz zu fahrzeugbezogenen IT-Systemen, wo zum heutigen Stand wenig 
relevante Ransomware-Angriffe bekannt sind, werden in anderen IT-Bereichen 
bereits erfolgreich Ransomware-Attacken durchgeführt. Egal ob öffentliche und 
private Unternehmens-IT-Systeme, industrielle Steuerungssoftware, Websites, 
Smartphones oder sogar Live-TV-Sender – jede Art von IT-Systemen kann und ist 
bereits Opfer eines Ransomware-Angriffs gewesen. 
\newline
Aktuelle Studien schätzen den Anteil an infizierten, unaufgeforderten Emails auf 
bis zu 70$\%$. Im Jahr 2016 haben so Cyberkriminelle bereits rund 1 Milliarde US-Dollar 
erpresst. 2021 wird der Umsatz auf bis zu 20 Milliarden US-Dollar ansteigen \cite[vgl.]{C..24.05.2020}.  
\newline
Heutige Ransomware macht sich die zunehmende Digitalisierung und Konnektivität aller 
Lebensbereiche sowie die wachsenden Abhängigkeit von vernetzten IT-Systemen zunutze
und genau dies könnte schon bald auch moderne Fahrzeuge betreffen. 
Fahrzeuge werden nämlich immer:

\begin{itemize}
    \item Softwaregesteuerter (Vergrößerung der Anzahl potenzieller Angriffsziele)
    \item Vernetzter (Vergrößerung der potenziellen Angriffsfläche)
    \item Komplexer (Erhöhung der ausnutzbaren Sicherheitslücken),
\end{itemize}
was die Anfälligkeit gegenüber Cybersicherheitsangriffen deutlich erhöht. 

In der Praxis wurden bereits alle bekannten Angriffsmuster hinsichtlich der 
Fahrzeugsicherheit (z.B. sicherheitskritische Fahrfunktionen wie Fahrzeuglenkung 
und -bremsung) erfolgreich demonstriert. Angriffe, welche sich jedoch auf die 
Fahrsicherheit des Fahrzeugs auswirken, noch nicht. Dies liegt daran, dass zur 
heutigen Zeit die häufigsten Fahrzeugsicherheitsangriffe immer noch dieselben sind 
wie früher: Fahrzeug-(Komponenten)Diebstahl, Kilometerzähler-Manipulation, (Chip-)Tuning 
und Herstellung gefälschter Teile. 
\newline
Gegen Diebstahl oder Phishing (Ausnutzen unvorsichtigen Verhaltens) entwickelt die 
Automobilindustrie immer wieder neue und verschiedene Maßnahmen und Mechanismen. 
Ransomware hingegen wurde von den Sicherheitsingenieuren laut der Autoren noch nicht wirklich 
in Angriff genommen. Dafür sehen sie hauptsächlich zwei Gründe:

\begin{enumerate}
    \item Die Erstellung eines Fahrsicherheitsangriffs erfordert viel Zeit und Geld (viele 
    Personenarbeitsmonate $\&$ \textgreater100.000 Dollar Entwicklungskosten), ermöglicht aber nur einen Angriff auf einen 
    bestimmten Fahrzeugtyp oder eine Fahrzeugklasse. 
    Eine Übertragung ist aufgrund der Homogenität der meisten Fahrzeug-IT-Architekturen und 
    Fahrzeug-IT-Softwaren nicht möglich.
    \item Der finanzielle Gewinn bei einem solchen Angriff bleibt bisher noch aus und oftmals 
    werden die Angreifer nur mit einer Anerkennung aus akademischen Kreisen belohnt. 
\end{enumerate}

Wie genau ein solcher Ransomware-Angriff in der Theorie und Praxis realisiert werden und 
aussehen könnte, wird in den nächsten Kapiteln beschrieben.
