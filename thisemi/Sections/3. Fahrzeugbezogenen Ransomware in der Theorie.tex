Da es nach aktuellem Stand noch keine öffentlich bekannten fahrzeugbezogenen 
Ransomware-Angriffe gibt, lassen sich die Voraussetzungen für solch eine Cyber-Attacke 
nur grob abschätzen und von anderen Ransomware-Angriffen im IT-Bereich übertragen. Damit 
jedoch ein derartiger Cyber-Angriff tatsächlich umgesetzt werden kann, bedarf es der 
Notwendigkeit mindestens folgender fünf Bedingungen:

\begin{enumerate}
    \item Einen Ransomware-Malware-Client und eine Server-Software für die On-Board-Realisierung 
    der Cyber-Erpressung auf dem Zielfahrzeug zusammen mit der entsprechenden Fernsteuerung.
    \item Ein anonymes Botnetz zur globalen Verteilung und Fernsteuerung der Ransomware-Fahrzeugclients.
    \item Ein fahrzeuginterner Sicherheits-Exploit meist zusammen mit Trojaner-Software zum 
    Erreichen und Infizieren einer angeschlossenen Fahrzeugeinheit, um den Ransomware-Malware-Client 
    zu installieren und auszuführen.
    \item Eine bordeigene Sperr- oder Bricking-Aktion für eine kritische Fahrzeugkomponente, die 
    nicht (leicht) wiederhergestellt oder umgangen werden kann oder die sich keine lange Ausfalldauer 
    leisten kann; idealerweise kombiniert mit einem (geheimen) Entsperrbefehl, um die gesperrte 
    Fahrzeugkomponente nach Zahlung des Lösegelds freizugeben.
    \item Ein anonymes Zahlungsschema zur Entgegennahme des Lösegelds und zum Schutz des Erpressers 
    vor Enttarnung und anschließenden rechtlichen Schritten. 
\end{enumerate}

Sind diese Voraussetzungen mindestens gegeben, dann könnte ein Ransomware-Angriff auf ein 
Fahrzeug wie folgt ablaufen.
\newline
Zu allererst muss der Angreifer die Schadsoftware erstellen. Sobald eine funktionierende 
Ransomware programmiert wurde, muss diese auf die ausgewählten Erpressungs-Zielfahrzeuge 
mittels einer Ransomware-Steuerungssoftware verteilt werden. Im besten Fall geschieht dies 
mittels eines anonymen Botnetzes, welches beispielsweise auf der TOR-Technologie 
(FUßZEILE: „The Onion Router“ – Verschlüsselung der Daten in mehreren Schichten, um anonymes 
surfen zu ermöglichen) basieren könnte. 
\newline
Hat die Software das Zielfahrzeug erreicht, muss dieses infiziert werden. Das kann auf Zwei 
wegen passieren: direkt und indirekt. 
\newline
Eine direkte Infizierung könnte beispielsweise über eine USB-Schnittstelle oder Ladeschnittstelle 
erfolgen. Eine indirekte Infizierung hingegen würde sich eine sekundäre Sicherheitslücke zu Nutze 
machen und über eine Zwischenkomponente versuchen, z.B. eine infizierte Website, auf welche das 
Fahrzeug zugreifet, suchen, die Software auf das Fahrzeug zu laden. 
\newline
Wenn dieses Vorgehen erfolgreich geklappt hat und die Software das Fahrzeug erreicht hat, wird 
der integrierte primäre Sicherheits-Exploit des Fahrzeuges genutzt, um den Ransomware-Client auf 
einer zentralen, gut vernetzten Einheit im Fahrzeug zu installieren und auszuführen. Diese Einheit 
könnte z.B. das Infotainment-System sein, welche dann als Host für weitere Aktionen missbraucht 
wird.
\newline
Je nachdem, mit welchem Vorgehen die Erpresser das Auto manipulieren wollen, kann der 
Ransomware-Client entweder eine Online-Verbindung zurück zum Erpresser aufbauen, um weitere 
Daten und/oder Befehle zu erhalten oder direkt über die fahrzeuginternen Bussysteme nutzen. 
Über diese könnte die Schadsoftware mit kritischen Steuergeräten (z.B. Motorsteuerung) 
kommunizieren, um somit die geplante Sperraktion durchzuführen. 
\newline
Hat dieses Vorhaben erfolgreich geklappt, dann muss nur noch die Erpressermeldung mit den 
nötigen Details zur anonymen Bezahlung auf einem Bildschirm im Fahrzeug angezeigt werden. 
Im Falle, dass das Opfer das geforderte Lösegeld tatsächlich bezahlt, würde die Ransomware 
erneut das Bot-Netzwerk kontaktieren, um den (geheimen) Entsperrbefehel/Entsperrungsbefehl? 
Zu erhalten und damit das Fahrzeug wieder freizugeben.
