Wie bereits in Kapitel 2 erwähnt ist fast jedes IT-System angreifbar 
für Ransomware und viele davon wurden bereits erfolgreich infiziert. 
Der dadurch entstandene lukrative Markt hat dazu geführt, dass es mittlerweile 
fertige Ransomware-Baukästen und Ransomware-as-a-Service (RaaS)-Angebote gibt. 
Solche Kits enthalten den notwendigen Malware-Kontrollserver und eine fertige 
Schnittstelle zu einen anonymen Bezahlungsmittel und Traffic-Anonymisierer. 
\newline
Einige solcher Ransomware-Kits bieten darüber hinaus auch gängige Sicherheits-Exploits 
für die Verteilung der Ransomware und Zielfunktion zur Verfügung. In einigen Fällen 
kann ein solches Kit sogar sogenannte „Zero-Day-Exploits“ zur Integration in die 
Software bereitstellen. Dabei handelt es sich um Sicherheitslücken, welche dem 
Entwickler/Unternehmen der betroffenen Einheit noch nicht bekannt ist. Dadurch 
kann diese Schwachstelle noch individueller und leistungsfähiger ausgenutzt werden.
\newline
Selbst der gewünschten Erpressungsmechanismus kann bei solchen Baukästen ausgewählt 
werden. Egal ob Verschlüsseln der Daten, Sperren wichtiger Komponenten oder mögliches 
Freigeben sicherheitskritischer Daten – der Erpresser hat fast unbegrenzte Möglichkeiten, 
den für sich optimalen Erpressungsmechanismus auszuwählen.
\newline
So kann mittels dieser Kits ein Angreifer innerhalb kurzer Zeit und mit wenigen Klicks 
eine funktionierende Ransomware mit allen notwendigen Komponenten erstellen.
\newline
Bis jetzt ist es jedoch noch nicht möglich, mit Hilfe solcher Baukästen Schadsoftware 
für fahrzeugbezogene IT-Systeme zu erstellen, da die meisten Kits hauptsächlich für 
Microsoft Windows-Betriebssysteme ausgelegt sind. Laut Autoren ist es jedoch nur eine 
Frage der Zeit und finanziellen Attraktivität, bis solche Erstellungssoftwares auch für 
\textit{automotive Linux} oder \textit{AUTOSAR-OS} programmiert werden. 
