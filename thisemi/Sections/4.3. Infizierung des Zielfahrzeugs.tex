Hat die Ransomware nun eine digitale Fahrzeugschnittstelle erreicht, muss der 
Client nur noch auf der entsprechend leistungsfähigen und gut vernetzten elektronischen 
Steuereinheit (ECU) im Fahrzeug installiert und ausgeführt werden.
\newline 
Mehrere Studien bzw. Forschungsarbeiten haben bereits bewiesen, dass fahrzeuginterne 
Sicherheitslücken existieren, welche durch eine Ransomware missbraucht werden könnten. 
In einer dieser gelang es den Sicherheitsexperten Zugriff auf die interne Software zu 
erlangen. Von Ärgernissen wie unkontrolliertem Hupen bis hin zu ernsthaften Gefahren 
wie dem Bremsen des Prius bei hohen Geschwindigkeiten, dem Abschalten der Servolenkung, 
dem Fälschen des GPS und Tachometer wird berichtet. Mit den Befehlen, welche die "Hacker" von ihren Laptops aus 
schickten, war fast alles möglich, um das Auto in eine gefährliche Fahrsituation zu 
bringen \cite[vgl.]{Greenberg.24.07.2013}.
\newline
Sowohl die zunehmende Digitalisierung, Vernetzung, Homogenisierung sowie Standardisierung 
im Fahrzeug tragen dazu bei, dass die Skalierbarkeit 
der Angriffe erhöht wird, da so mehr einheitliche Fahrzeugsicherheitsschwachstellen 
entstehen. Diese wären:

\begin{itemize}
    \item Schwachstellen im USB-Anschluss des Fahrzeug-Infotainment-Systems
    \item Schwachstellen im OBD-Port für den Zugriff auf alle Busse im Fahrzeug
    \item CD/DVD-Player-Schwachstellen am Fahrzeug-Infotainment-System
    \item Bluetooth-Pufferüberlauf-Schwachstellen bei Fahrzeug-Infotainment-Einheit
    \item Zellulare Verwundbarkeit an der zentralen Fahrzeugkommunikationseinheit
    \item Wi-Fi-Schwachstellen am Ladesystem für Elektrofahrzeuge
    \item Remote-Schwachstellen am Telematik-Steuergerät (TCU) des Nachrüstmarktes
    \item Schwachstellen in mobilen Fahrzeug-Apps für den Zugriff auf Fahrzeuginterna
    \item Ausnutzung des Wi-Fi-Stacks durch Google Project Zero
\end{itemize}
	
