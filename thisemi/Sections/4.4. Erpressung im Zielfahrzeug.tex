Wurde die Ransomware erfolgreich installiert und ausgeführt, kann die 
eigentliche Geiselnahme beginnen. In diesem Fall ist die Geisel sinnbildlich:

\begin{itemize}
    \item eine gesperrte oder "gemauerte" kritische Fahrzeugkomponente, die nicht 
    (einfach) wiederhergestellt oder umgangen werden kann oder die sich keine lange 
    Ausfalldauer leisten kann
    \item die Beschlagnahmung oder das Durchsickern kritischer Fahrzeugdaten, die nicht 
    (einfach) wiederhergestellt werden können oder die einen erheblichen Schaden 
    verursachen würden, wenn sie öffentlich zugänglich würden
    \item irgendetwas anderes, um das Opfer zur Zahlung des Lösegelds zu zwingen 
    (z.B. reine Behauptung, es sei etwas gesperrt worden)
\end{itemize}

Sobald einmal der Zugriff gelungen ist, hat der Angreifer schier unendliche 
Möglichkeiten, um seine Epressungsaktion durchzuführen. In Abbildung \ref{fig:Erpressung} sind 
mögliche Systemkomponenten dargestellt, welche sich zum Bricken, Sperren oder 
Leaken anbieten würden.
\newline

\begin{figure}[htbp!]
    \centering
    \includegraphics[width=0.94\textwidth]{Images/Diagnostic-Scanning-1.jpg}
    \caption{Mögliche Systemkomponenten, welche durch eine Ransomware missbraucht werden könnten}
    \label{fig:Erpressung}
\end{figure}

Abbildung \ref{fig:Erpressung} macht deutlich, dass es in heutigen Autos zahlreiche Systeme gibt, 
auf denen man ein Cyber-Angriff durchführen kann. Sei es das Blockieren wichtiger 
Steuergerätsfunktionen oder kryptografischer Anmeldeinformationen, das Verschlüsseln 
von kritischen Daten, das Freigeben kritischer (interner) Daten im Fahrzeug, das 
Manipulieren von Sensor- oder Servodaten oder das Manipulieren/Zerstören kritischer 
Fahrzeugkomponenten – ein Angreifer hat schier unendliche Möglichkeiten. 
\newline
Sobald die Geiselnahme durchgeführt (oder vorgetäuscht) wurde, wird die Ransomware 
für das Opfer sichtbar. Eine solche Erpressungsmeldung erklärt in der Regel die eigentliche 
Erpressungssituation klar und deutlich und bietet sehr detaillierte Hilfe und Informationen, 
wie das geforderte Lösegeld zu zahlen ist. Zur Demonstration haben die Autoren eine 
solche Meldung erstellt, welche in Abbildung \ref{fig:WannaDrive} zu sehen ist. 
\newline

\begin{figure}[htbp!]
    \centering
    \includegraphics[width=0.94\textwidth]{Images/WannaDrive.png}
    \caption{Beispielhafte Erpressungsmeldung mit Zahlungsinformationen}
    \label{fig:WannaDrive}
\end{figure}
