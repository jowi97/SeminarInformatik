Wenn man die ganze Situation etwas genauer betrachtet, dann zeichnet sich ab, dass dieses 
Vorgehen eher für gewerbliche und öffentliche Fahrzeuge geeignet ist. In den meisten Fällen 
würde eine Privatperson keine absolute Dringlichkeit haben, das Auto sofort wieder in einen 
funktionierenden Zustand zu bringen, sondern eher Werkstattexperten zur Hilfe holen und 
keineswegs große Lösegeldsummen zahlen. 
\newline
Unternehmen mit großen Fahrzeugflotten sind hingegen vielversprechende Opfer, da diese auf 
ihre Fahrzeuge angewiesen sind und bei einem Ausfall mit großen Verlusten zu rechen ist. 
Dementsprechend stellen folgenden Fahrzeugtypen deutlich lukrativere „Lösegeldopfer“ dar:

\begin{itemize}
    \item Lkw mit ihren engen Zeitplänen, empfindlichen Gütern und hohen Vertragsstrafen
    \item Reisebusse mit ihren ähnlich engen Zeitplänen, die bis zu 70 Passagiere in Eile transportieren
    \item Landmaschinen, die Millionen \$ kosten, aber nur einige Wochen im Jahr auf dem Feld eingesetzt werden
    \item Baufahrzeuge oder andere Spezialfahrzeuge mit komplexer, teurer und gefährlicher Ausrüstung
    \item Behördenfahrzeuge, die für die öffentliche Sicherheit wichtig sind, wie z.B. Polizei-, Feuerwehr- oder Krankenhausfahrzeuge
    \item Autovermietungen oder Autoleasingfirmen, aber auch große firmeneigene Fahrzeugflotten oder ein Fahrzeughersteller selbst
    \item und sogar Militärfahrzeuge
\end{itemize}

Angenommen, eines der oben genannten Fahrzeuge wäre infiziert und das Opfer ist bereit, 
das Lösegeld zu zahlen, ist es von essentieller Bedeutung, dass der Ransomware-Client einen 
schnellen, einfachen, benutzerfreundlichen und anonymen Zahlungskanal bereitstellt. Am besten 
eignet sich eine kryptische Währung wie z.B. Bitcoin oder Ethereum. Mittlerweile existieren 
Smartphone-Apps, welche praktisch alle physischen Währungen in anonyme Kryptowährungen 
umwandeln und transferieren.
\newline
Damit das Fahrzeug nach einer erfolgreichen Bezahlung wieder freigegeben werden kann, muss 
beim Bezahlvorgang eine eindeutige Fahrzeugkennung erstellt werden. Nur so ist es am Ende 
für den Erpresser möglich zu erkennen, welches Fahrzeug wieder entsperrt werden muss. Eine 
solche Identität kann in das Zahlungsschema eingebunden werden, indem jedes Fahrzeug eine 
individuelle Zahlungszieladresse besitzt oder der Ransomware-Client sendet eine verschlüsselte 
individuelle Nachricht an seinen Bot-Master.
\newline
Ist eine Zahlung beim Erpresser eingegangen, dann muss dieser nochmals eine Verbindung mit 
dem entsprechenden Ransomware-Client aufnehmen, um diesem den Entsperrbefehl zu senden und 
das Fahrzeug (im besten Fall) wieder freizugeben.  
