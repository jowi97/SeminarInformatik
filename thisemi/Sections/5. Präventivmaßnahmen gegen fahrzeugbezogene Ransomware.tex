Wie bereits erläutert, gibt es unzählige Möglichkeiten, ein 
Fahrzeug mittels Ransomware zu attackieren. Dementsprechend gibt es auch keine 
allumfassende Schutzmaßnahme. Vielmehr benötigt man ein 
ganzheitliches Security-Engineering-Konzept, das einen vollständigen, systematischen 
und mehrschichtigen Schutzansatz verfolgt. 
Dieses Konzept umfasst:

\begin{itemize}
    \item Das komplettes Fahrzeugsystem (d.h. vom einzelnen Steuergerät bis 
    zum angeschlossenen Cloud-Backend)
    \item Den gesamten Fahrzeuglebenszyklus (d.h. von der ersten Anforderungsanalyse 
    bis zur Ausmusterung des Fahrzeugs)
    \item Die komplette Fahrzeugorganisation (d.h. von den Sicherheitsprozessen 
    bis zur Security Governance)
\end{itemize}

Dies wiederum gestaltet sich als schwierig und kostspielig im Gegensatz zu 
klassischen IT-Systemen, da Fahrzeuge deutlich mehr Angriffspunkte haben, 
keine effektive Sicherung von Daten oder Funktionen durchführen können, in den 
meisten Fällen keine regelmäßigen Sicherheitsupdates erhalten und nur eine einfache 
Firewall besitzen.
