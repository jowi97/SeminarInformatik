Da man davon ausgehen muss, dass ein Angreifer das Zielfahrzeug nach der 
schwächsten Komponente absuchen würde, muss bei der Absicherung das gesamte 
Fahrzeug betrachtet werden. Das heißt, dass man von jedem einzelnen Steuergerät 
bis hin zu angeschlossenen Backend-Diensten alles betrachtet. 
\newline
Des Weiteren ist es von Vorteil, gleichzeitig mehrere Verteidigungslinien zu haben, 
da man davon ausgehen muss, dass eine der Schutzmaßnahmen geschwächt wird oder ausfällt.
\newline
Einen Schutzansatz nach dem „Singel Point of Failure“ Prinzip, sollte man definitiv 
vermeiden. Dieser geht davon aus, dass eine einzige Komponente (z.B. Firewall) das 
gesamte sichere interne Fahrzeugnetzwerk von einem unsicheren, externen Netzwerk isoliert. 
Sollte ein Angreifer in dieser Komponente eine Schwachstelle entdecken, hätte das zur 
Folge, dass auf einmal alle Fahrzeuge des betroffenen Typs komplett kompromittiert würden. 
\newline
Um ein solches Szenario zu verhindern und entgegenzuwirken muss der Angriffspfad so 
oft wie möglich durchbrochen werden. Dazu empfehlen die Autoren folgende Maßnahmen:

\begin{itemize}
    \item Fahrzeug-Cybersicherheit Intelligenz $\&$ Forschung
    \item Klassische Unternehmenssicherheit für alle Fahrzeug-IT-Infrastrukturen
    \item Starke Backend-Zugriffskontrolle für alle fahrzeugbezogenen Assets, Schnittstellen 
    und Funktionalitäten
    \item Vollständiger Schutz der Fahrzeugschnittstellen
    \item Sichere Fahrzeug-E/E-Architektur
    \item System zur Erkennung und Verhinderung von Eindringlingen in das Fahrzeug
    \item Fahrzeuginterne Firewall
    \item Verfahren zur Reaktion auf Vorfälle
\end{itemize}
