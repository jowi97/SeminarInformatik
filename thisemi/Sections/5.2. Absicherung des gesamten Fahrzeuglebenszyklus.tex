Neben der Absicherung des gesamten Fahrzeugsystems muss auch der 
gesamte Fahrzeuglebenszyklus betrachtet werden. Dieser erstreckt sich 
vom Start der Entwicklung bis zur Ausmusterung des Fahrzeugs. Dies garantiert, 
dass man auf die sich ständig ändernden Sicherheitsanforderungen schnellstmöglich 
reagieren kann und somit neu entdeckte Schwachstellen beheben und neu entwickelte 
Sicherheitsansätze einbringen kann.
\newline
Ein solcher kontinuierlicher Lebenszyklus hat auch einige zusätzliche technische 
und organisatorische Implikationen. Die gesamte Entwicklungshardware, alle 
Werkzeugketten und zumindest ein Teil der beteiligten Experten müssen bis zur endgültigen 
Ausmusterung verfügbar bleiben, was für schwere Nutzfahrzeuge einen Zeitraum
von bis zu 20 Jahren bedeutet.
