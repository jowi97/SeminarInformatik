Ist der Fall eingetreten, dass man Opfer eines Ransomware-Angriffs geworden ist, empfehlen Experten, 
das Lösegeld auf keinen Fall zu bezahlen. Oft erfolgt, trotz Zahlung des geforderten Lösegeldes, keine 
Entsperrung des Fahrzeugs. Ebenso wird verhindert, dass dieses Vorgehen Nachahmer fördert. 
\newline
Um aber nun das Fahrzeug bzw. die Fahrzeugflotte wieder in einen fahrtauglichen Zustand zu versetzen und 
somit größere finanzielle Schäden zu verhindern, empfehlen die Autoren einen neunstufigen Notfallplan. 
Unter der Voraussetzung, dass ein erfahrenes Fahrzeugsicherheitsexpertenteam, ein fahrzeugspezifischer 
Sicherheits-Notfallplan und eine Remote-Software-Update-Funktionalität vorhanden sind, wird folgendes 
Notfallverfahren empfohlen:

\begin{enumerate}
    \item Möglichst frühzeitige Erkennung von Fahrzeug-Ransomware durch Meldungen erster Opfer 
    oder durch das Cyber-Security-Intelligence Team des Unternehmens
    \item Analysen von Fahrzeug-Ransomware durch Fahrzeugsicherheitsexperten
    \item Sicherheitsrisiko- und Schwachstellenbewertung durch Fahrzeugsicherheitsexperten auf Basis 
    einer systematischen Auswertung für entsprechende Angriffspotenziale 
    \item Identifizierung und Bewertung möglicher Gegenmaßnahmen und Reaktionen, einschließlich der 
    Zahlung des geforderten Lösegelds
    \item Entscheidung der Unternehmensleitung auf Basis der Vorschläge der Sicherheitsexperten für 
    Reaktionsmaßnahmen (z. B. technische und nicht-technische Maßnahmen), Incident Response Communication, 
    weitere Risikovorsorge, etc.
    \item Entwicklung, Test und Vorbereitung von:
    \begin{enumerate}
        \item Technischen Reaktionsmaßnahmen wie z.B. die Bereitstellung von Steuergeräte-Backup-Firm"-ware, 
        Fahrzeug- und/oder Infrastruktur-Software-(Sicherheits-)Patches, Steuergeräte-/Fahr"-zeug-
        Neukonfigurationsbefehle
        \item Nicht-technischen Reaktionsmaßnahmen wie Angreifer-Diplomatie, Lösegeldzahlung, Informieren 
        allgemeiner Cyber-Abwehrbehörden
        \item Incident Response-Kommunikation innerhalb des Unternehmens und gegenüber Kunden, Lieferanten, 
        Partnern, Branchengremien, Behörden oder Medien
    \end{enumerate}

    \item Roll-out von technischen und nicht-technischen Reaktionsmaßnahmen und Beginn der Incident-Response-
    Kommunikation
    \item Kontinuierliche Überwachung und (Neu-)Bewertung von Auswirkungen, Risiken und Erfolg aller 
    durchgeführten Reaktionsmaßnahmen zur Anpassung von Maßnahmen
    \item Untersuchung und Sicherheitsforensik zur Dokumentation und Berichterstattung, vor allem aber zur 
    langfristigen Eindämmung, vollständigen Wiederherstellung, Lessons-Learned, Präventions"-maßnahmen und 
    aktualisierten Überwachung.
\end{enumerate}