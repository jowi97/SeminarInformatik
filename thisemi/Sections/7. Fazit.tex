Aktuell sind Ransomware-Angriffe auf Fahrzeuge noch Theorie, doch schon in naher 
Zukunft kann sich dies ändern. Aufgrund der zunehmenden Digitalisierung und Konnektivität 
von Autos bieten sich den Angreifern viele Angriffsflächen, welche, wenn die Automobilindustrie 
ihren Securityansatz nicht um den Aspekt Ransomware erweitert, schon bald rigoros ausgenutzt 
werden könnten. Hinter Ransomware-Angriffen steckt ein überzeugendes Geschäftsmodell und wenn 
man dadurch viel Geld verdienen kann, wird es auf kurze oder lange Sicht Personen geben, die 
sich das zum Vorteil machen möchten.
\newline
Opfer werden weniger Privatpersonen sein, sondern eher Nutzfahrzeuge und große Fahrzeugflotten, 
da diese meist auf ihre Fahrzeuge angewiesen sind und sich keine lange Ausfalldauer leisten können.
\newline
Es ist unabdingbar, dass sich die Automobilindustrie auf Ransomware-Angriffe vorbereiten muss, 
indem sie die Schutzfunktionen im gleichem Tempo wie deren Vernetzung und Digitalisierung
auf Fahrzeuge ausweitet, sich mit ganzheitlichen, 
mehrschichtigen Schutzmaßnahmen auf die Bedrohung vorbereitet, aber auch die Fähigkeit erlangt, mit 
aktualisierten Abwehrmaßnahmen und Reaktionen auf Angriffe zu reagieren.
\newline
Dieses Paper gibt nur einen kleinen Einblick in das Thema fahrzeugbezogene Ransomware-Angriffe. 
Aufgrund der begrenzten Seitenanzahl konnten nicht alle Aspekte erwähnt oder tiefgründiger diskutiert 
werden, aber das Themengebiet bietet viel Potential, in welchem tiefgründigere Forschungen 
getätigt werden können, um somit der Sicherheit für Autos beizutragen. 
